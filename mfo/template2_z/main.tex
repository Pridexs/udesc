\documentclass{article}
\usepackage[utf8]{inputenc}
\usepackage{zed-csp}
\usepackage{bbm}
\usepackage{fancyhdr}
\usepackage{graphicx}
\usepackage[left=3.00cm, right=2.00cm, top=5.00cm, bottom=3.00cm]{geometry}

\thispagestyle{fancy}
%\fancyhead[R]{Universidade do Estado de Santa Catarina/Centro de Ciências Tecnológicas UDESC/CCT}
%\fancyhead[L]{}

\lhead{
	\includegraphics[scale=0.75]{figuras/logo_udescjlle.pdf}
}
\chead{
	\scriptsize{
		UNIVERSIDADE DO ESTADO DE SANTA CATARINA -- UDESC\\
		CENTRO DE CIÊNCIAS TECNOLÓGICAS -- CCT\\
		DEPARTAMENTO DE CIÊNCIA DA COMPUTAÇÃO -- DCC
	}
}
%\rhead{\includegraphics[width=0.3\columnwidth]{figuras/logo_udescjlle.png}}
\rhead{
	\includegraphics[scale=0.75]{figuras/logo_dcc.pdf}
}


\begin{document}

\begin{center}
    \huge  Operações com cartão de crédito\\
\end{center}

\begin{center}
    Nome Sobrenome\\
    Métodos Formais - Linguagem Z\\
\end{center}
\vspace{1.5cm}

Modelagem de um sistema de cartão de crédito, incluindo suas operações como:
\begin{itemize}
  \item Autenticação (falha/sucesso)
  \item Efetuar pagamento (falha/sucesso)
  \item Verificação de saldo (falha/sucesso)
  \item Pagamento com crédito (falha/sucesso)
  \item Pagamento limite (falha/sucesso)\\
  \\
\end{itemize}

\textbf{Tipos globais da especificação :}

\begin{zed}
    [CartãoDeCrédito] \\
   Caracter : \{a, b, c, ..., z\}\\
   Retorno == Falha na Autenticação | Saldo Insuficiente | Limite Estourado | OK\\
\end{zed}

\begin{schema}{CartãoDeCrédito}
    cliente: {\it seq1}\ Caracter \\
    senha: \nat \\
    numeroCartão:\nat\\
    saldo: \real\\
    limiteUsado:\real\\
    limiteTotal:\real
    
    \where
    
    \#senha=6\\
    limite \geq 0\\
    \#numeroCartão=16\\
    limiteUsado \leq limiteTotal\\
    saldo \geq 0\\
    limiteUsado \geq 0
\end{schema}

\pagebreak

Para a autenticação ser efetuada com sucesso, o número de cartão precisa existir em CartãoDeCrédito e a senha fornecida precisa ser a mesma contida em CartãoDeCrédito.\\
Caso positivo, a mensagem de OK será mostrada.
\begin{schema}{Autenticação}
    \Xi CartãoDeCrédito\\
    numeroCartão?: \nat\\
    senha?: \nat\\
    msg!:Retorno
\where
    numeroCartão? \in CartãoDeCrédito \land numeroCartão?.senha = senha?\\
    msg! = OK
    
\end{schema}

\begin{schema}{FalhaAutenticação}
    \Xi Autenticação\\
    \Xi CartãoDeCrédito\\
    numeroCartão?: \nat\\
    senha?: \nat\\
    msg!:Retorno
\where
    (numeroCartão? \in CartãoDeCrédito \land numeroCartão?.senha \neq senha?) \lor\\
    (numeroCartão? \nin CartãoDeCrédito)\\
    msg! = Falha na Autenticação
    
\end{schema}

Para EfetuarPagamento ser executada com sucesso, o saldo do cartão precisa ser maior ou igual ao valor do pagamento. Com isso, o novo saldo será calculado subtraindo-se o valor pago.\\
Caso positivo, a mensagem de OK será mostrada.

\begin{schema}{EfetuarPagamento}
    \Delta CartãoDeCrédito\\
    \Xi Autenticação\\
    numeroCartão?: \nat\\
    senha?: \nat\\
    valor?: \real
    msg!: Retorno

    \where
    \Xi Autenticação\\
    saldo \geq valor?\\
    saldo' = saldo - valor?\\
    msg! = OK
    
\end{schema}

\begin{schema}{FalhaEfetuarPagamento}
    \Xi EfetuarPagamento\\
    \Xi CartãoDeCrédito\\
    numeroCartão?: \nat\\
    senha?: \nat\\
    valor?: \real
    msg!: Retorno

    \where
    \Xi FalhaAutenticação \lor saldo \leq valor?\\
    \IF saldo \leq valor? \THEN msg! = Saldo Insuficiente
    
\end{schema}

Para verificar o saldo, a autenticação deve ser executada com sucesso. Caso positivo, duas mensagens serão mostradas, a de saldo e o limite disponível.

\begin{schema}{VerificarSaldo}
    \Xi CartãoDeCrédito\\
    \Xi Autenticação\\
    limite:\real \\
    numeroCartão?: \nat\\
    senha?: \nat\\
    msg1!: \real\\
    msg2!: \real
    
    \where
    
    \Xi Autenticação\\
    msg1!=saldo\\
    msg2!=limiteTotal - limiteUsado
\end{schema}

Caso a verificação falhe, a mensagem de FalhanaAutenticação será mostrada:

\begin{schema}{FalhaVerificarSaldo}
    \Xi VerificarSaldo\\
    \Xi CartãoDeCrédito\\
    saldo: \real\\
    numeroCartão?: \nat\\
    senha?: \nat\\
    msg!: Retorno
    
    \where
    
    \Xi FalhaAutenticação\\
    msg! = Falha na Autenticação
\end{schema}

\pagebreak

Para o pagamento ser efetuado com sucesso, o valor a ser pago + limite usado deve ser menor que o limite total. Será tambem atualizado o valor do limiteUsado.\\
Caso positivo, a mensagem de OK será mostrada.

\begin{schema}{EfetuarPagamentoCrédito}
    \Delta CartãoDeCrédito\\
    \Xi Verificação\\
    numeroCartão?: \nat\\
    senha?: \nat\\
    valor?: \real\\
    msg!: Retorno

    \where
    \Xi Autenticação\\
    limiteTotal \geq valor? + limiteUsado\\
    limiteUsado' = limiteUsado + valor?\\
    msg! = OK
    
\end{schema}

\begin{schema}{FalhaEfetuarPagamentoCrédito}
    \Xi CartãoDeCrédito\\
    \Xi EfetuarPagamentoCrédito\\
    numeroCartão?: \nat\\
    senha?: \nat\\
    valor?: \real\\
    msg!: Retorno

    \where
    \Xi FalhaAutenticação \lor (limiteTotal < limiteUsado + valor?) \\
    \IF limiteTotal < (limiteUsado + valor?) \THEN msg!= Limite Estourado
    
\end{schema}

\pagebreak

Para EfetuarPagamentoLimite ser executado com sucesso, o saldo deve ser maior ou igual ao valor a ser pago. Será atualizado o valor do limiteUsado.\\
Caso positivo, a mensagem de OK será mostrada.

\begin{schema}{EfetuarPagamentoLimite}
    \Delta CartãoDeCrédito\\
    \Xi Autenticação\\
    numeroCartão?: \nat\\
    senha?: \nat\\
    valor?: \real\\
    msg!: Retorno

    \where
    \Xi Autenticação\\
    saldo \geq valor?\\
    limiteUsado' = limiteUsado - valor?\\
    msg! = OK
    
\end{schema}

Caso o saldo seja inferior ao valor necessário para o pagamento, a mensagem de SaldoInsuficiente será mostrada:

\begin{schema}{FalhaEfetuarPagamentoLimite}
    \Xi CartãoDeCrédito\\
    \Xi EfetuarPagamentoLimite\\
    numeroCartão?: \nat\\
    senha?: \nat\\
    valor?: \real\\
    msg!: Retorno

    \where
   \Xi FalhaAutenticação \lor (saldo \leq valor?)\\
   \IF saldo \leq valor? \THEN msg!= Saldo Insuficiente
    
\end{schema}

\end{document}
