\documentclass{article}
\usepackage[utf8]{inputenc}
\usepackage{zed-csp}
\usepackage{bbm}
\usepackage{fancyhdr}
\usepackage{graphicx}
\usepackage[left=3.00cm, right=2.00cm, top=5.00cm, bottom=3.00cm]{geometry}

\thispagestyle{fancy}
%\fancyhead[R]{Universidade do Estado de Santa Catarina/Centro de Ciências Tecnológicas UDESC/CCT}
%\fancyhead[L]{}

\lhead{
	\includegraphics[scale=0.75]{figuras/logo_udescjlle.pdf}
}
\chead{
	\scriptsize{
		UNIVERSIDADE DO ESTADO DE SANTA CATARINA -- UDESC\\
		CENTRO DE CIÊNCIAS TECNOLÓGICAS -- CCT\\
		DEPARTAMENTO DE CIÊNCIA DA COMPUTAÇÃO -- DCC
	}
}
%\rhead{\includegraphics[width=0.3\columnwidth]{figuras/logo_udescjlle.png}}
\rhead{
	\includegraphics[scale=0.75]{figuras/logo_dcc.pdf}
}


\begin{document}

\begin{center}
    \huge  Exercício Tanque 1\\
\end{center}

\begin{center}
    Alexandre Maros\\
    Métodos Formais - Linguagem Z\\
\end{center}
\vspace{1.5cm}

Resolução do Exercício 1 - Tanque\\\\

\textbf{Tipos globais da especificação :}

\begin{zed}
   mensagens ::= qtd\_minima\_OK | qtd\_minima\_NOK | qtd\_maxima\_OK | qtd\_minima\_NOK\\
   luzes ::= verde | vermelho\\
   alarmes :== ligado | desligado
\end{zed}

\textbf{a --} Encher o tanque até o máximo permitido\\
\begin{schema}{EncherTanqueMax}
    \Delta tanqueCombustivel\\
    qtdC: \nat
    
    \where
    
    qtdC=nivel\_max - conteudo\\
    \heq nivel\_min \leq conteudo + qtdC \leq nivel\_max\\
    conteudo' = conteudo + qtdC
\end{schema}

\textbf{b --} Encher o tanque com qualquer quantidade, respeitando o limite máximo
\begin{schema}{EncherTanque}
    \Delta tanqueCombustivel\\
    qtdC?: \nat
    
    \where
    
    conteudo + qtdC? \leq nivel\_max\\
    conteudo' = conteudo + qtdC?
\end{schema}

\textbf{c --} Indicar transbordamento
\begin{schema}{MonitorarMax}
    \Xi tanqueCombustivel\\
    luz: luzes\\
    alarme: alarmes\\
    msg: mensagens
    
    \where
    
    conteudo > nivel\_max\\
    luz = vermelha\\
    alarme = ligado\\
    msg != qtd\_maxima\_NOK
\end{schema}

\pagebreak

\textbf{d --} Esvaziar o tanque por uma quantidade qualquer, respeitando o mínimo
\begin{schema}{EsvaziarTanque}
    \Delta tanqueCombustivel\\
    qtdC?: \nat
    
    \where
    
    conteudo - qtdC? \geq nivel\_min\\
    conteudo' = conteudo - qtdC?
\end{schema}

\textbf{e --} Esvaziar completamente o tanque
\begin{schema}{EsvaziarCompletamente}
    \Delta tanqueCombustivel
    
    \where
    
    conteudo' = 0
\end{schema}

\textbf{f --} Esvaziar o tanque para uma quantidade que estoura o limite mínimo e avisa este ocorrido
\begin{schema}{EsvaziarTanqueEstourandoMinimo}
    \Delta tanqueCombustivel\\
    qtdC?: \nat\\
    msg: mensagens
    
    \where
    
    0 \leq conteudo - qtdC < nivel\_min\\
    conteudo' = conteudo - qtdC?\\
    msg != qtd\_minima\_NOK
\end{schema}

\end{document}
