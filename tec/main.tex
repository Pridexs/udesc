\documentclass{beamer}

\usepackage{graphicx,hyperref,udesc,url}
\usepackage[utf8]{inputenc}
\usepackage[T1]{fontenc}
\usepackage{booktabs}
\usepackage[portuges]{babel}


\title[Teoremas da Incompletude de Gödel]{Os Teoremas da Incompletude de Gödel e suas implicações}

\author[Alexandre M, Nadyan P.]{
    Alexandre Maros, Nadyan S. Pscheidt\\\medskip
    {\small \url{alehstk@gmail.com}} \\
{\small \url{nadyan.suriel@gmail.com}}}

\institute[UDESC]{
    Departamento de Ci\^encia da Computa\c{c}\~ao \\
    Centro de Ci\^encias e Tecnol\'ogias\\
Universidade do Estado de Santa Catarina}

\begin{document}

\begin{frame}
    \titlepage

\end{frame}

%
% TOPICOS E CONCEITOS
%

\section{Tópicos e Conceitos}
\begin{frame}
    \frametitle{Tópicos}

    \begin{itemize}
        \item Conceitos
        \item O Primeiro Teorema da Incompletude
        \item O Segundo Teorema da Incompletude
        \item Implicações
        \item Conclusão
    \end{itemize}
\end{frame}

\begin{frame}
    \frametitle{Conceitos}

    \begin{itemize}
        \item \textit{\textbf{Sistema Formal}}: Um sistema de axiomas equipados com regras de inferência, que possibilitam a criação de novos teoremas.
        \item Um sistema formal é \textbf{completo} se, para toda afirmação da linguagem do sistema, ou essa afirmação ou a negação dessa afirmação são derivadas no sistema.
        \item Um sistema formal é \textbf{consistente} se o sistema não tem contradições dentro de si mesmo.
        \begin{itemize}
            \item i.e, \textit{"Esta frase não é provável em $F$"}
        \end{itemize}
    \end{itemize}
\end{frame}

%
% O 1o TEOREMA
%

\section{O Primeiro Teorema}
\begin{frame}
    \frametitle{O Primeiro Teorema de Gödel - Definiçao}

    \begin{center}
        \textit{"Qualquer teoria efetivamente gerada capaz de expressar aritmética elementar não pode ser tanto \textbf{consistente} quanto \textbf{completa}."} \\~\\~\\
        \textit{"[...] existe uma afirmação aritmética que é verdadeira, mas que não pode ser provada em teoria."}
    \end{center}
\end{frame}

\begin{frame}
    \frametitle{Prova}

    \begin{itemize}
        \item Consistência-$\omega$
        \item Representabilidade
        \item Mapeamento da Linguagem Formal (\textbf{Números de Gödel})
        \item Lema da diagonal
    \end{itemize}
\end{frame}

\begin{frame}
    \frametitle{Prova Alternativa}

    \begin{itemize}
        \item[] Suponha que temos um sistema formal $\textbf{F}$ completo e consistente que seja capaz de entender máquinas de Turing.
        \item[] Usando F nós podemos resolver o problema da parada da seguinte forma:
            \begin{itemize}
                \item Dado uma máquina M, nos queremos saber se ela para em uma fita branca.
                \item Logo, enumeramos todas as possíveis provas de F até achar uma que diz que a máquina \textit{para} ou que entra em \textit{loop}.
            \end{itemize}
        \item[] Por o sistema ser \textbf{completo}, iremos achar tal prova.
        \item[] Como o sistema é \textbf{consistente}, eu posso ter certeza dessa prova.
        \item[] Logo $F$ não pode existir pois estamos resolvendo o problema da Parada.
    \end{itemize}

\end{frame}

%
% O 2o TEOREMA
%

\section{O Segundo Teorema}
\begin{frame}
    \frametitle{O Segundo Teorema de Gödel - Definição}

    \begin{center}
        \textit{"Para todo sistema consistente F com um certo conjunto aritmético elementar, a consistência de F \textbf{\textit{não}} pode ser provada no próprio F."}
    \end{center}
\end{frame}

\begin{frame}
    \frametitle{O Segundo Teorema de Gödel - Consistência}

    Sendo:
    \begin{center}
        \begin{itemize}
            \item $cons(F)$ a consistência de $F$;
            \item $G_F$ a sentença para $F$;
        \end{itemize}
    \end{center}
    
    %vdash quer dizer "satisfaz"
    Temos que:
    \begin{center}
        $F \vdash cons(F) \rightarrow G_F$ \\
    \end{center} 
    Ou seja, se $cons(F)$ fosse provável em $F$, $G_F$ também seria, porém contradizendo o primeiro teorema. Então $cons(F)$ \textbf{não} pode ser provável em $F$.\\
    
    Então:
    \begin{center}
        $F \nvdash cons(F)$
    \end{center}
\end{frame}

%
% EFEITOS
%

\section{Implicações}
\begin{frame}
    \frametitle{Trabalhos inspirados}

    Certamente os teoremas de Gödel inspiraram importantes pesquisadores e seus trabalhos, alguns deles são:
    \begin{itemize}
        \item Alonzo Church e seus estudos na área de indecidibilidade, 1936.
        \item Martin Hugo Löb e seus estudos sobre demonstrabilidade, do tipo \textit{"Se uma fórmula F for demonstrável, então certamente F é verdadeira"}, 1955.
    \end{itemize}
\end{frame}

\begin{frame}
    \frametitle{Efeitos em outras áreas}

    Outras áreas de pesquisa foram afetadas pelos teoremas, como:
    \begin{itemize}
        \item Filosofia da matemática, onde alguns acreditam que os teoremas de Gödel refutam o logicismo.
        %logicismo acredita que as formulações lógicas são suficientes por si mesmas, assim como toda a matematica pode ser redutivel à logica.
        \item Física teórica, onde a afirmação de que uma teoria não pode ser capaz de provar tudo e a si mesma sem conceitos externos pode impossibilitar a existência da tão buscada "teoria de tudo". 
    \end{itemize}
\end{frame}

%
% CONCLUSAO MAYBE???
%

\section{Conclusão}
\begin{frame}
    \frametitle{Conclusão}

    Os Teoremas da Incompletude de Gödel foram de fato importantes para definir o rumo das ciências na época.\\
    
    Os conceitos descritos e provados transformaram o modo de muitos pesquisadores olharem para seus campos de estudo, além de inspirar muitos outros a alcançarem as provas de mais teoremas baseados em suas fórmulas complexas
    
\end{frame}

\begin{frame}{Referências}

    \begin{itemize}
        \item P. Raatikainen, \textbf{Goodel’s incompleteness theorem}s, in:  E. N. Zalta (Ed.), The Stanford Encyclopedia of Philosophy, spring 2015 Edition, 2015.
    \end{itemize}
     
\end{frame}

\end{document}

