Em geral, podemos ver que é possível ordenar arquivos grandes o suficiente tal que a memória não consiga comportar. Para isso dispomos de vários métodos de classificação e intercalação, aqui utilizamos a classificação pelo método de seleção por substituição e o balanceamento de n caminhos.
\par
O tempo total do ordenamento se dá por diversos fatores, sendo eles tempo de escrita e leitura do disco, tempo de processamento das informações na memória, criação de novos arquivos, abertura de arquivo, alocação de informação, são alguns exemplos. Escrever no disco rígido leva mais tempo do que indexar e trabalhar com registros na memória, vimos que com um arquivo muito grande, não é possivel carregar todos os registros na memória, ordenar, e salvar novamente, para que essa tarefa seja possível é necessário muitas operações de escrita e leitura do disco.
\par
Abordamos aqui um dos métodos clássicos para ordenar uma quantidade enorme de dados, atualmente temos uma infinidade de outros métodos para realizar este tipo de tarefa. 

